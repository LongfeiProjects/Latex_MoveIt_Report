\subsection{MS5-6 Rev2c possible improvements}
Possible improvements on the MS5-6~Rev2c:
%\subsubsection{Possible improvements}
\begin{itemize}
	\item Torque compensator would benefit of more tuning in regards to the former stability analysis.
	\item Firmware development for new outputs and faster refresh rates would help to perform frequency responses.
	\item PID would benefit of more tuning, namely by performing frequency responses.
	\item PID could be replaced by a more general compensator similar to the one used on torque.
\end{itemize} 

\subsection{MS7}
MS7 arms will have a different architecture compared to MS5-6~Rev2c. Therefore, work done on Rev2c may no be valuable for MS7. Major differences, which address a lot of limitations of Rev2c arms, are:
\begin{itemize}
	\item Second encoder on the output size may allow vibration damping. If this is the case, torque feedback compensation might be no longer needed.
	\item Torque loop will be ran at 1~kHz.
	\item Use of current, velocity and position loops.
	\item More rigid links.
	\item Faster chips and better communication will allow faster data acquisition of any needed data.
\end{itemize}


\subsection{Future works}
Future works currently in scope for vibration damping and controller loops tuning are: 
\begin{itemize}
	\item Development of standard methods for controller tuning and stability analysis for MS7.
	\item Give specific requirements to firmware team in terms of loops design, rates and datalogging for MS7.
	\item Perform better characterization of Rev2c actuators and arms and develop a simulation model.
	\item Update of the simulation model to MS7 specifications to speed up controller design.
\end{itemize}

%\subsubsection{Design changes} 
%The current torque feedback filter has been fixed for MS 5-6. However design improvements can continue for research purposes which might
%be useful for MS 7. It is to be noted however that, MS 7 will have very different control architechture than MS 5-6 and results will not 
%directly feed into it.


%\subsection{MS 7}
%\begin{itemize}
%	\item With the encoder on the output side, vibrations due to the harmonic drive can possibily be damped by the position loop itself.
%	If so, torque feedback compensation might be no longer needed.  
%	\item Simulations - MS 7 hardware will not be available for near future. Simulating Harmonic Drive vibrations, with encoder on the output side
%	will help speed up control design and stability analysis. 
%\end{itemize}